\documentclass[11pt,]{article}
\usepackage{lmodern}
\usepackage{amssymb,amsmath}
\usepackage{ifxetex,ifluatex}
\usepackage{fixltx2e} % provides \textsubscript
\ifnum 0\ifxetex 1\fi\ifluatex 1\fi=0 % if pdftex
  \usepackage[T1]{fontenc}
  \usepackage[utf8]{inputenc}
\else % if luatex or xelatex
  \ifxetex
    \usepackage{mathspec}
  \else
    \usepackage{fontspec}
  \fi
  \defaultfontfeatures{Ligatures=TeX,Scale=MatchLowercase}
\fi
% use upquote if available, for straight quotes in verbatim environments
\IfFileExists{upquote.sty}{\usepackage{upquote}}{}
% use microtype if available
\IfFileExists{microtype.sty}{%
\usepackage{microtype}
\UseMicrotypeSet[protrusion]{basicmath} % disable protrusion for tt fonts
}{}
\usepackage[margin=1.5cm]{geometry}
\usepackage{hyperref}
\hypersetup{unicode=true,
            pdfborder={0 0 0},
            breaklinks=true}
\urlstyle{same}  % don't use monospace font for urls
\usepackage{graphicx,grffile}
\makeatletter
\def\maxwidth{\ifdim\Gin@nat@width>\linewidth\linewidth\else\Gin@nat@width\fi}
\def\maxheight{\ifdim\Gin@nat@height>\textheight\textheight\else\Gin@nat@height\fi}
\makeatother
% Scale images if necessary, so that they will not overflow the page
% margins by default, and it is still possible to overwrite the defaults
% using explicit options in \includegraphics[width, height, ...]{}
\setkeys{Gin}{width=\maxwidth,height=\maxheight,keepaspectratio}
\IfFileExists{parskip.sty}{%
\usepackage{parskip}
}{% else
\setlength{\parindent}{0pt}
\setlength{\parskip}{6pt plus 2pt minus 1pt}
}
\setlength{\emergencystretch}{3em}  % prevent overfull lines
\providecommand{\tightlist}{%
  \setlength{\itemsep}{0pt}\setlength{\parskip}{0pt}}
\setcounter{secnumdepth}{0}

%%% Use protect on footnotes to avoid problems with footnotes in titles
\let\rmarkdownfootnote\footnote%
\def\footnote{\protect\rmarkdownfootnote}

%%% Change title format to be more compact
\usepackage{titling}

% Create subtitle command for use in maketitle
\newcommand{\subtitle}[1]{
  \posttitle{
    \begin{center}\large#1\end{center}
    }
}

\setlength{\droptitle}{-2em}

  \title{}
    \pretitle{\vspace{\droptitle}}
  \posttitle{}
    \author{}
    \preauthor{}\postauthor{}
    \date{}
    \predate{}\postdate{}
  
\usepackage{titlesec}
\titlespacing*{\section}{0pt}{0.75\baselineskip}{0.1mm}
\usepackage[table]{xcolor}
\usepackage{tikz}
\usetikzlibrary{positioning,shapes}
\usepackage{fancyhdr}
\pagestyle{fancy}
\fancyhf{}
\cfoot{\thepage}
\renewcommand{\headrulewidth}{0pt}
\chead{\begin{tikzpicture}\fill[blue] (current page.north west) rectangle ([yshift = -1cm]current page.north east);\fill[gold] ([yshift = -1.15cm]current page.north west) rectangle ([yshift = -1cm]current page.north east);\end{tikzpicture}}
\hypersetup{colorlinks=true, linkcolor=blue, urlcolor=blue}
\let\oldhref\href
\renewcommand{\href}[2]{\oldhref{#1}{\bfseries#2}}

\begin{document}

 \renewcommand{\familydefault}{\sfdefault}

\tikzset{%
  missing/.style = {minimum width = 0.64\textwidth, minimum height = 0.2\textheight, font = \large},%
  present/.style = {minimum width = 0.64\textwidth},%
  every picture/.style = {remember picture, overlay, inner sep = 0pt, line width=0pt,/utils/exec={\sffamily}}}

\definecolor{blue}{HTML}{003D7E} \definecolor{gold}{HTML}{FCCA06}
\definecolor{grey}{HTML}{505050}

\def\UrlFont{\bfseries}

\begin{tikzpicture}

  
  \fill[blue] (current page.north west) rectangle ([yshift = -3cm]current page.north east);
  \fill[gold] ([yshift = -3cm]current page.north west) rectangle ([yshift = -3.15cm]current page.north east);
  
  
  \node[below = 1cm of current page.north, white, font = \Huge\bf] (num) {Aquifer Factsheet - Companion Document};
  \node[below = 0.3cm of num, white, font = \large\bf] {Updated 2018-11-15};

\end{tikzpicture}

\vspace{1.5cm}

The purpose of this companion document is to provide more detailed
information related to the terms, analytical methods, and data sources
used to produce the Aquifer factsheets.

\subsubsection{\texorpdfstring{\emph{Disclaimer}}{Disclaimer}}\label{disclaimer}

\vspace{-4mm}

The information in this Factsheet has been prepared from information
currently available to the BC government. As available information is
limited in nature, this Factsheet only provides a broad overview of
information about the aquifer and is not intended to provide a
comprehensive description of the aquifer. The Factsheet is being
provided as a public service on an ``as is'' basis and without any
warranty as to the fitness or suitability of the information in it for
any particular purpose. The information in this Factsheet has not been
tested or verified by the BC government. Consequently, this Factsheet
should not be relied upon as providing complete or specific information
or advice for use in responding to or assessing particular sites or
circumstances.

Persons using this Factsheet should take steps to independently verify
the information herein provided. It is your responsibility to review
your particular sites or circumstances and then to determine the
accuracy, suitability, reliability, usability, completeness, timeliness
or applicability of the information in this Factsheet to your particular
sites or circumstances. Provision of this information does not
substitute for identification and analysis of individual sites or
circumstances by a suitably qualified professional.

Persons using this Factsheet do so at their own risk. The BC government
accepts no liability or responsibility for the relevancy, suitability,
reliability, usability, completeness, timeliness, accuracy or
applicability of any of the information in this Factsheet to individual
sites or circumstances, nor for any results obtained from its use. Users
of this Factsheet are wholly responsible for independently verifying the
appropriateness of the information herein provided to their individual
sites or circumstances based on their own professional advice.

\begin{center}\rule{0.5\linewidth}{\linethickness}\end{center}

\section{Aquifer Description (based on
Subtype)}\label{aquifer-description-based-on-subtype}

The aquifer description is a generic description based on the subtype
classification of the aquifer assigned at the time of mapping. A
complete list of Aquifer subtype code descriptions can be found on the
\href{https://www2.gov.bc.ca/gov/content/environment/air-land-water/water/groundwater-wells/aquifers/aquifer-subtype-code-description}{BC
Government ground water website}.

\url{https://www2.gov.bc.ca/gov/content/environment/air-land-water/water/groundwater-wells/aquifers/aquifer-subtype-code-description}

\section{Water District}\label{water-district}

British Columbia is divided into named and described water districts. A
complete list of the Water Districts is provided in the
\href{http://www.bclaws.ca/civix/document/id/complete/statreg/38_2016}{Water
Sustainability Act}.

\url{http://www.bclaws.ca/civix/document/id/complete/statreg/38_2016}

\section{No. of Wells Correlated to
Aquifers}\label{no.-of-wells-correlated-to-aquifers}

The number of wells correlated to an aquifer represents the number of
wells that were identified as being completed within the aquifer at the
time of mapping. There are several reasons why wells appearing within
the aquifer polygon may not be correlated to the aquifer including: a)
the well was constructed after the time of mapping, b) no lithological
information provided on the well log, or c) the well was completed
either below or above the identified aquifer unit.

\section{Vulnerability, Productivity and Aquifer
classification}\label{vulnerability-productivity-and-aquifer-classification}

Values are determined at the time of mapping according to the BC Aquifer
Classification system, which are described in
\href{http://www.env.gov.bc.ca/wsd/plan_protect_sustain/groundwater/aquifers/reports/aquifer_maps.pdf}{The
Guide to using the BC Aquifer Classification Maps -- For the Protection
and Management of Groundwater}. The guide presents a detailed
description of the BC Aquifer Classification System, as well as the
methodologies employed in classifying the aquifer, and discussions on
some of the limitations of the data. This should help the reader to
better understand the criteria used to identify, delineate and classify
an aquifer. The classification component characterizes the aquifer based
on the level of development of the groundwater resource (the water
supply available relative to the amount of demand placed on that water
supply) at the time of mapping, and also based on the vulnerability of
the aquifer to contamination.

\url{http://www.env.gov.bc.ca/wsd/plan_protect_sustain/groundwater/aquifers/reports/aquifer_maps.pdf}

\section{Hydraulic Connectivity}\label{hydraulic-connectivity}

The likelihood of hydraulic connectivity is inferred based on aquifer
sub-type. The determination is based on a desktop assessment and has not
been field tested or verified. The guidance document
\href{http://a100.gov.bc.ca/appsdata/acat/documents/r50832/HydraulicConnectMW3_1474311684426_4310694949.pdf}{Determining
the Likelihood of Hydraulic Connection} provides detailed information
related to the methodologies and assumptions underlying the assessment.

\url{http://a100.gov.bc.ca/appsdata/acat/documents/r50832/HydraulicConnectMW3_1474311684426_4310694949.pdf}

\section{BC Aquifer Stress Index}\label{bc-aquifer-stress-index}

The aquifer stress index uses the groundwater footprint and
aquifer-scale estimates of withdrawal, recharge and the groundwater
contribution to environmental flows. A description of the data sources
and the methods and limitations is available on the BC Aquifer Stress
index website.

Aquifers are classified as:

\begin{itemize}
\tightlist
\item
  More stressed (highly certain) if ALL results suggest aquifer stress
\item
  More stressed (less certain) if SOME results suggest aquifer stress
\item
  Less stressed if NONE of the results suggest aquifer stress
\item
  Methods not applicable where data is not available or for confined
  aquifers
\end{itemize}

\href{http://governmentofbc.maps.arcgis.com/home/webmap/viewer.html?webmap=6c137fb01a364ee699440a28619e45c2}{http://governmentofbc.maps.arcgis.com/home/webmap/viewer.html?\\
webmap=6c137fb01a364ee699440a28619e45c2}

\section{Reported Well Yield}\label{reported-well-yield}

Reported well yields are based on the estimates recorded by the driller
at the time of well construction. It is only an estimate and is not
necessarily based on measured values. As defined in
\href{http://www.env.gov.bc.ca/wsd/plan_protect_sustain/groundwater/aquifers/reports/aquifer_maps.pdf}{The
Guide to using the BC Aquifer Classification Maps -- For the Protection
and Management of Groundwater}, estimated yields are categorize as
follows:

\begin{itemize}
\tightlist
\item
  \textless{} 0.3 L/s are considered ``Low'' yield
\item
  0.3 - 3.0 L/s are considered ``Medium'' yield
\item
  \textgreater{} 3.0 L/s are considered ``High'' yield
\end{itemize}

\url{http://www.env.gov.bc.ca/wsd/plan_protect_sustain/groundwater/aquifers/reports/aquifer_maps.pdf}

\section{Reported Static Water Depth}\label{reported-static-water-depth}

Water depth is reported in meters below ground surface and represents
the water level measured by the driller at the time of well
construction.

\section{Reported Well Depth}\label{reported-well-depth}

The well depth is reported in meters below ground surface as measured by
the driller at the time of well construction.

\vspace{5mm}

\begin{minipage}[t]{0.6\textwidth}
  \section{Summary Box Plots for Well Yield, Static Water Depth and Well Depth}
  \vspace{3mm}
  The boxplots provide a summary of the available information for wells correlated to the aquifer.\\
  
  \begin{itemize}
    \item n = the number of wells included in the summary
    \item upper 'hinge' = first quartile
    \item lower 'hinge' = third quartile
    \item crossbar = median
    \item whisker = maximum and minimum values\\(within interquartile range)
    \item points = outliers (defined as 1.5 x interquartile range)
  \end{itemize}
  
  \vspace{0.5cm}
  
  \textbf{Note:} Data not plotted when n < 5
  
  \vspace{0.5cm}
  
\end{minipage}\begin{minipage}[t]{0.4\textwidth}
  \centering
  \textbf{Box Plot Key}\\
  \includegraphics[width = 0.7\textwidth]{../out/boxplot_key.png}
\end{minipage}

\section{Water Level Summary and Precipitation Summary
Plot}\label{water-level-summary-and-precipitation-summary-plot}

The Water Level Summary Plots are based on data collected through the
Provincial Groundwater Observation Well Network.

\url{https://catalogue.data.gov.bc.ca/dataset/57c55f10-cf8e-40bb-aae0-2eff311f1685}

The dataset comprises daily mean water levels collected from all
provincial observation wells. Historically, the water level data was
collected at varying frequencies and with measurement frequencies that
have changed over the years. To summarize the data, the median values
were calculated for any month that has more than one reading. The median
and percentile values plotted on the graph were then calculated based on
monthly values for all years available. Monthly water level summaries
were produced for observation wells with a minimum of 10 years of data.
``Preliminary'' plots are provided for data sets with between 5 and 10
years of data. The extreme Minimum and Maximum values represent the
highest and lowest values ever recorded for a particular month.

The Precipitation Summary Plots are based on the
\href{http://climate.weather.gc.ca/climate_normals}{Canadian Climate
Normals (1981-2010) dataset accessible from the Environment Canada
website}. The nearest climate station at similar elevation to the
aquifer (and typically within a maximum distance of 10km) was selected
as representative.

\url{http://climate.weather.gc.ca/climate_normals}

\section{Trend Analysis}\label{trend-analysis}

The trend analysis was conducted by Environmental Reporting BC. A full
description of the methods used for the trends analysis are available on
the
\href{http://www.env.gov.bc.ca/soe/indicators/water/groundwater-levels.html}{State
of The Environment website}.

\url{http://www.env.gov.bc.ca/soe/indicators/water/groundwater-levels.html}

\section{Piper Plot}\label{piper-plot}

The chemistry results presented in the Piper plot data were obtained
from the Environmental Monitoring System (EMS) web reporting site
(\url{https://a100.gov.bc.ca/pub/ems}) using the R package ``rems''
(\url{https://github.com/bcgov/rems/blob/master/README.Rmd}) and
converted to a format compatible with the groundwater software program
Aquachem (Waterloo Hydrogeologic, 2014). Chemistry results were screened
for data quality by assessing the charge balance error. Only samples
with a charge balance error of less than 10\% were included on the piper
plot.

\textbf{Reference}\\
Waterloo Hydrogeologic (2014). AquaChem Water Quality Analysis \&
Geochemical Modeling v.2014.2.


\end{document}
